\documentclass[a4paper,11pt,draft]{scrartcl}
% \usepackage[hmargin=4.3cm,vmargin=0.5cm]{geometry}
\usepackage{amsmath}
\usepackage{amsfonts}
\usepackage{amssymb}
\usepackage{microtype}
\DeclareMathOperator*{\argmax}{argmax}
% \newcommand{\argmax}[1]{\arg\underset{#1}{\max}\;}
\renewcommand{\equiv}{\Leftrightarrow}
\newcommand{\vnorm}[1]{\left|\left|#1\right|\right|}
\begin{document}
\section*{Setup}
We have two cities, $C_1$ and $C_2$. In each of them lie objects of different
kind (in our case: photos, check-ins and venues). These objects have a
position in $\mathbb{R}^2$ along with other attributes, like time, tags,
category or number of visitors.

\section*{Goal}
As an input, we are given a polygonal region $R$ contained in $C_1$
% (this could be rectangles in the beginning and later more complicated, yet
% reasonable, shape)
. The goal is then to find the region $R'$ in $C_2$ that
is the most similar to $R$.
% (with a potential warning if no suitable candidate is found).
Now let us see \emph{one} way to turn this objective into a problem.

\section*{Formalization}
First we need a function that transforms a region $R$, defined by all the $e$
objects it contains, into a numeric representation. In the simplest case, we
aggregate information from these $e$ objects into a single $d$ dimensional
feature vector
\begin{align*}
	\phi \colon \mathcal{R} &\to \mathbb{R}^d \\
	R &\mapsto x
\end{align*}
A less appealing alternative would be to keep track of the $e$ object and having
a representation in $\mathbb{R}^{e\times d}$. But I am not sure this choice is
actually part of the problem.

Then we need a function, parametrized by $\theta$, to asses the similarity
between two regions in different cities
\[
	\sigma_{\theta} \colon \mathbb{R}^d \times \mathbb{R}^d \to [0, K]
\]

For instance, a simple class of similarity could be
\vspace{-0.3\baselineskip}\[
    \sigma_{\theta} (x_1, x_2) = (x_1 - x_2)^T \theta (x_1 - x_2)
\vspace{-0.3\baselineskip} \]
where $\theta$ is a positive definite matrix.

\medskip

At this stage, we can fulfill our objective by computing, for a given $R \in
\mathcal{R}(C_1)$
\begin{equation}
    \argmax_{R' \in \mathcal{R}(C_2)}\; \sigma_{\theta}
    \left(\phi(R), \phi(R')\right)
    \label{e:onetheta}
\end{equation}
A more refined answer would acknowledge that not all regions can be compared in
the same way and thus solve
\begin{equation}
    \argmax_{\theta ,\, R' \in \mathcal{R}(C_2)}\; \sigma_{\theta}
    \left(\phi(R), \phi(R')\right) - cost(\theta)
    \label{e:manythetas}
\end{equation}
where the regularisation of $\theta$ ensures that there is still a common
ground in the way similarity is computed.

\bigskip

We are almost ready to formulate the problem. Namely, let assume we have a way
to evaluate the quality of matching (either by ground truth or from some
generic criteria). We want to find the best $\phi$, $\sigma$ (and $\theta$ if
we use formulation~\eqref{e:onetheta}, in which $\theta$ is a global
parameter). A subproblem may be to perform these computations efficiently.

\end{document}
