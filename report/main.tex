\begin{abstract}
Provide a short description of the project (problem studied, contributions, findings).
\end{abstract}

\section{Introduction}

A common way of organizing information on websites it to rely on tags provided
by users. Although these words are chosen freely, we do not expect them to be
arbitrary and thus hope to be able to extract limited semantics from them. In
this project, we focus on tags associated with geolocated photos extracted
from
Flickr\footnote{\href{https://secure.flickr.com/}{http://www.flickr.com/}}.
Using this crowdsourced set of (tags, location) pairs, we investigate three
problems:
\begin{itemize}
	\item Given a tag, find the places in a city where it is significantly
		concentrated.
	\item Conversely, given a location, find the tags that best describe it
		compared with other locations of similar scale.
	\item Finally, find the places in the city that seems to generate the
		most interest.
\end{itemize}

In the first case, it could be applied in a touristic context. When arriving in a
new city, a person interested in \textsf{baseball} or \textsf{streetart} could
find relevant places according to the experience of inhabitants and previous
visitors. The second answer could be used by Flickr to suggest relevant tags
when users upload photos. Consequently, more sensibly tagged photos allow more
relevant search result and improve the website user experience. The last could
for instance the first step of an automatic travel guide that select point of
interest in a city or a region.

Other applications may include inferring missing information (like location or
time of the day) from the tags or…

What are the contributions of this project?

\section{Setting}
\label{sec:setting}

Provide information that will be necessary for a reader to understand your contributions.

\subsection{General Description of the Dataset}

Using the Flickr API, we downloaded metadata from every photos satisfying a
set of criteria: they contained at least one tag, they were located, they have
been uploaded after January 1\textsuperscript{st}, 2008 and they belonged to a
predefined rectangular region. Most of the work was done on a set of around
\numprint{780000} photos in the city of San Francisco, but we also got data in
a part California\footnote{from San José to Reno} and over the whole United
States.

More precisely, in addition to tags and location, we know when each photos was
taken and uploaded, by which user and what title was given to it (the title
was not used except when it contained hashtag, that were converted to tags).
Thus a typical data point look like this:

\vspace{\baselineskip}
{\centering\begin{minipage}{0.7\linewidth}
  \begin{Verbatim}[frame=none, gobble=0]
loc     : [-122.392501, 37.77515],
taken   : "2008-03-24 14:55:40",
user_id : "37417902@N00",
tags    : ["sanfrancisco", "california", "bridge", "chinabasin"]
title   : "sf 4th st bridge 8"
  \end{Verbatim}
\end{minipage}\par}

\subsection{General Description of the Dataset}

Using the Flickr API, we downloaded metadata from every photos satisfying a
set of criteria: they contained at least one tag, they were located, they have
been uploaded after January 1\textsuperscript{st}, 2008 and they belonged to a
predefined rectangular region. Most of the work was done on a set of around
\numprint{780000} photos in the city of San Francisco, but we also got data in
a part California\footnote{from San José to Reno} and over the whole United
States.

More precisely, in addition to tags and location, we know when each photos was
taken and uploaded, by which user and what title was given to it (the title
was not used except when it contained hashtag, that were converted to tags).
Thus a typical data point look like this:

\vspace{\baselineskip}
{\centering\begin{minipage}{0.7\linewidth}
  \begin{Verbatim}[frame=none, gobble=0]
loc     : [-122.392501, 37.77515],
taken   : "2008-03-24 14:55:40",
user_id : "37417902@N00",
tags    : ["sanfrancisco", "california", "bridge", "chinabasin"]
title   : "sf 4th st bridge 8"
  \end{Verbatim}
\end{minipage}\par}

\subsection{Uncertainty of the data}
\label{p:data}
\begin{itemize}
	\item User id are subject to caution since nothing prevent people to
 upload photos on behalf of others. It would require serious effort to
 detect it but one may expect it is rather uncommon. Moreover, the mere
 fact that the upload take place still denotes a relation between the
 user and the photos.
	\item While timestamp issued by mobile phones are likely to be correct,
 as their internal clock is synchronized by internet, this may not
 always be the case for dedicated cameras. More concerning than usual
 drift of low quality clock is the situation of tourists coming from
 different timezone. Yet as I could not think of any simple solution to
 that problem, I just ignored it and carried on.
	\item To ensure the quality of the localization, I restrict myself to
 photos whose precision is deemed \enquote{street level} by Flickr. The
 potential problem is that it would again cost an extra request to know
 whether this location was given by GPS (in which case the camera
 position is accurate) or by the user at upload time. In the latter
 case, in addition to the general imprecision of the method, it is
 ambiguous whether this location refer the place where the photo was
 shot or the position of the photo's subject\footnote{Think of a bridge
 taken from a nearby hill.}.
	\item Finally, without additional request, the tags obtained are those
 normalized by Flickr. This normalization is not bijective but it is
 assumed that two tags with the same normalized form were close in the
 first place.
\end{itemize}

Overall, these restrictions are not really problematic. Yet there is another
one that is not specific to a given field. Users have the possibility to
upload photos by batch and assign them common location and tags. In some case,
this could skew the corresponding distributions. Take the tag
\textsf{14thstreet} as an example. One user have uploaded more than
\numprint{3500} photos at one corner street during a marathon whereas only a
handful of others users have employed this tag, which is therefore not as
popular as the raw number would suggest. To alleviate this situation, I
performed a simple preprocessing step that I will describe later.

First, I duplicated the \enquote{tags} field of every photos. Then, for each
user $u$ and each tag $t$, I computed the distribution of photos tagged $t$ by
$u$ and removed the tags from the photos that appeared more than $T=120$ times
in the same place (the same cell of the $200\time 200$ discrete grid) in the same
time (2 weeks interval). With such threshold, it removed around 20\% of all
tags and it somewhat modified the list of top tag but with no clear pattern.
But tags with very low entropy like \textsf{14thstreet} did not appear
anymore because they lost most of their support. A better way to deal with
that issue could be to weight tags by the number of their users but it would
computationally more expensive.


\subsection{Statistical exploration}
\begin{figure}[hbtp]
\includegraphics[width=\columnwidth]{distag}
\caption{Tag distribution log-log scale over three regions of different scale.\label{f:tags}}
\end{figure}

The first thing was to look at the tags to get a sense of them. In San
Francisco after the preprocessing, there was a total of \numprint{4977625}
occurrences of the \numprint{145242} unique tags. But their distribution varies
widely, between the most popular one, \textsf{sanfrancisco}, used
\numprint{373427} times and the \numprint{101361} that are used less than 5
times. Some of them are shown in Table~\vref{tab:tags} but a more synthetic
visualization is presented in Figure~\vref{f:tags}, where we can see that like
words in a written documents, tags follows a power law.

\begin{table}[ht]
	\centering
	\begin{tabular}{lll}
		\toprule
		First 15 tags 	 & between 100 and 1000 & after \numprint{90000} \\
		\midrule
		sanfrancisco 	 & 2013 							   & sfgiantsfan \\
		california       & pacific                             & rolexbigboatseries \\
		iphoneography    & february                            & proshowgold \\
		square           & foundinsf                           & neutraface \\
		% squareformat     & dolorespark                         & natur€ \\
		instagramapp     & japaneseteagarden                   & lusty \\
		unitedstates     & boat                                & lightousetender \\
		sf               & 5k                                  & jennyholzer \\
		usa              & national                            & img0562jpg \\
		% ca               & σανφρανσίσκο                        & djguyruben \\
		san              & cruise                              & cutebaby \\
		francisco        & above                               & cardamine \\
		goldengatepark   & july2009                            & californiaproduce \\
		2010             & effortlesslyuploadedbymyeyeficard   & aroundwithb1 \\
		iphone           & dayofdecision                       & aquateenhungerforcemooninite \\
		\bottomrule
	\end{tabular}
	\caption{A sample of San Francisco tags, depending of their rank.\label{tab:tags}}
\end{table}

After making these observations, I decided to consider only tags with enough
support, both to ease the computational effort and to avoid outliers. For each
tags, I compute three simple metrics, total count, distinct users count and
time span. Using the three threshold (150 photos, 25 users, 500 days), I kept
only \numprint{1874} tags. It may seem quite restrictive but they still cover
68.6\% of all occurrences and we can always change these thresholds
later\footnote{For instance, with $(20, 2, 0)$, we get \numprint{12959} tags
and 84.4\% coverage.}.

We can then conduct a similar analysis over the locations in which photos
appear. Because of their large number, it was not convenient nor readable to
display them individually. Therefore, I discretized space as a regular grid of
size 200 by 200. In that case, each rectangular cell is around $80\times 70$
meters and I used this same method for all other spatial computation. A
natural way of visualizing them is to draw a heatmap (Figure~\vref{f:heat}).
We notice again that they are far from being uniformly distributed and that
some neighborhood are more popular than others. More quantitatively, number of
photos of each location as a function of their rank (Figure~\vref{f:pdis}), we
notice that it first follows a power law and after some point, a more
abrupt one. Moreover, the same phenomena occurs for other grid size, albeit
with different $\alpha$ coefficient. Despite this similar behavior, it was
more tricky to explicitly exclude parts of the city.

\begin{figure}[hbtp]
\includegraphics[width=\columnwidth]{../heatmap.png}
\caption{Photos count in logarithmic scale (the darker, the more
photos).\label{f:heat}}
\end{figure}

\begin{figure}[hbtp]
\includegraphics[width=\columnwidth]{../prez/pdistrib.png}
\caption{Spatial distribution of photos over three grids with different
granularity.\label{f:pdis}}
\end{figure}

Let us return to one of our original problem, find which tags describe a given
location. The first approach would to filter this \numprint{1874} tags to keep
only those that are enough concentrated at one position and reject those that
too uniformily distributed. After that, it would simply a matter of returning
those that appear in the place of interest. An example of this two kind of
tags are \textsf{museum} and \textsf{street}. As shown in Figure~\vref{f:sm},
\textsf{museum} photos are mostly located around five or six points whereas
\textsf{street} is more diffuse. But instead of looking at a map, we want a
numerical statistic that allow us to distinguish between this two cases.

\begin{figure}[hbtp]
\includegraphics[width=\columnwidth]{../prez/sm.png}
\caption{Red dots denote photos tagged \textsf{museum} while blue ones are
	\textsf{street}.\label{f:sm}}
\end{figure}

\section{Algorithms}
\label{sec:algorithms}

Define the algorithms you employ for the purposes of the project.

\section{Experimental Evaluation}

Here you report your findings from running your algorithms
(Section~\ref{sec:algorithms}) on your data (Section~\ref{sec:setting}).

\section{Conclusion \& Future Work}

\balance{}
By \keyfinding{summarizing} what we have done so far, we will be able to see
which \keyfinding{limitations} can be addressed in further work. After
extracting photos metadata from Flickr and exploring their location and their
tags, we computed some statistics about them and notably the discrepancy of
the most supported tags. We then use this information to highlight various
relationships between tags and locations. But how can these results can be
improved?

\medskip

The first point to look at is \keyfinding{the data}. As mentioned on
page~\pageref{p:data}, they come with some noise and we expect the results to
be more relevant if we manage to \keyfinding{clean them}. Yet it remains to be
seen if it is worth the extra effort. A simpler direction would be to
\keyfinding{increase their amount}. This can be done by crawling photos from
others cities, to see if the same methods produce the same results or if
working only with San Francisco have introduced some bias. Another approach
would be to use more diverse sources like instagram, foursquare or any other
social networks that allows users to share localized and tagged contents.


The second limitation that arise from focusing on a single area is that we
were not confronted enough with one fundamental issue of spatial analysis, the
\keyfinding{modifiable areal unit problem}\cite{scale}, which states that
statistical measurements can be profoundly affected by the choice of the
spatial partition and its characteristics size (in our case, the grid and its
cells dimension).  It is an open question whether the current approach can
simply be tuned to tackle different area like states, countries or even the
whole world, or if it needs radical change and scale invariant statistics.

A related issue is that the presented approach comes with \keyfinding{many
parameters and thresholds}. That would be fine if they encode some prior
knowledge but the truth is, they were chosen empirically. Indeed, the problem
is mostly unsupervised. For instance, except for specific tag like the name of
buildings, most of them can be found in several locations. Likewise, places
can be described by many tags, maybe of various relevance but with no clear
cut. Finally, finding interesting locations is rather subjective and,
depending of the chosen criterion, can yield diverse results. The rather rigid
grid division is therefore quite limited, not least because not all locations
are rectangular! It may thus be more appropriate to try some
\keyfinding{unsupervised clustering methods}, for instance based on the
Euclidean norm in the case of the photos locations or drawn from natural
language \keyfinding{topic modeling} approaches in the case of
tags\cite{topicModel}.


Even with this lack of definitive results, it would be helpful to have a way
to \keyfinding{evaluate them}. One way could be to generate
\keyfinding{artificial data} from which we know precisely what should be found
inside, but it would again require some assumptions of our part. We can also
create \keyfinding{hand picked ground truth} by manually annotating tags and
locations, as in \cite{Rattenbury2009}. To cope with ambiguity, it would be
even better to aggregate evaluations from several individuals\footnote{But
also more time consuming, although this may be presented as a game, like
GeoGuessr: \url{http://geoguessr.com/}.}. An easier alternative would be to
manually extract famous landmarks from existing tourist guides and assess
precision and recall of the method.

\medskip

Finally, whereas photos come with \keyfinding{four kinds of
information}—tags, location, user and time—we only work with the first two
and it would undoubtedly be more interesting to \keyfinding{use all of them}.
Let us try to list systematically what we can do of it.

First, by focusing on a single one and building a profile using the rest, we
may devise a \keyfinding{similarity measure and use it to perform clustering}.
In the case of users, it may divide between wealthy and not, young and old,
male or female \cite{gender}, tourist or inhabitant or richer category like
family with kids and traveler of a package tour. As said above, tags can be
grouped into common topic\footnote{Or at first, we can look at
co-occurrences.}. Lastly, although times and locations are subset of
$\mathbb{R}$ and $\mathbb{R}^2$ and as such already have a natural metric,
they carry additional semantic that we would like to exploit. For instance, we
assume that by periodicity, Sundays in 2008 and 2013 share common patterns
although they are far apart in time. Likewise, maybe locations in front of the
water or that consist of park have similar characteristics.

Then we should consider \keyfinding{pair of concepts}. The present work deals
with tags \texrel locations in both directions, thus there are $\binom{4}{2} -
1$ others to look for: tags \texrel time, tags \texrel users, locations \texrel
times, locations \texrel users and times \texrel users. Yet all of them may
not be interesting and we need to hierarchize our priorities. 

