The first thing was to look at the tags to get a sense of them. In San
Francisco after the preprocessing, there was a total of \numprint{4977625}
occurrences of the \numprint{145242} unique tags. But their distribution varies
widely, between the most popular one, \textsf{sanfrancisco}, used
\numprint{373427} times and the \numprint{101361} one that are used less than 5
times. Some of them are shown in Table~\vref{tab:tags} but a more synthetic
visualization is presented in Figure~\vref{f:tags}, where we can see that like
words in a written documents, tags follows a power law.

\begin{table}[ht]
	\centering
	\begin{tabular}{lll}
		\toprule
		First 15 tags 	 & between 100 and 1000 & after \numprint{90000} \\
		\midrule
		sanfrancisco 	 & 2013 							   & sfgiantsfan \\
		california       & pacific                             & rolexbigboatseries \\
		iphoneography    & february                            & proshowgold \\
		square           & foundinsf                           & neutraface \\
		squareformat     & dolorespark                         & natur€ \\
		instagramapp     & japaneseteagarden                   & lusty \\
		unitedstates     & boat                                & lightousetender \\
		sf               & 5k                                  & jennyholzer \\
		usa              & national                            & img0562jpg \\
		ca               & σανφρανσίσκο                        & djguyruben \\
		san              & cruise                              & cutebaby \\
		francisco        & above                               & cardamine \\
		goldengatepark   & july2009                            & californiaproduce \\
		2010             & effortlesslyuploadedbymyeyeficard   & aroundwithb1 \\
		iphone           & dayofdecision                       & aquateenhungerforcemooninite \\
		\bottomrule
	\end{tabular}
	\caption{A sample of San Francisco tags, depending of their rank.\label{tab:tags}}
\end{table}

After making these observations, we decided to consider only tags with enough
support, both to ease the computational effort and to avoid outliers. For each
tags, we computed three simple metrics: total count, distinct users count and
time span. Using the three thresholds (150 photos, 25 users, 500 days), we
kept only \numprint{1874} tags. It may seem quite restrictive but they still
cover 68.6\% of all occurrences and these thresholds can be changed
later\footnote{For instance, with $(20, 2, 0)$, we get \numprint{12959} tags
and 84.4\% coverage.}.

We can then conduct a similar analysis over the locations in which photos
appear. Because of their large number, it was not convenient nor readable to
display them individually. Therefore, we discretized space as a regular grid
of size 200 by 200. A natural way of visualizing them is to draw a heatmap
(Figure~\vref{f:heat}).  We notice again that they are far from being
uniformly distributed and that some neighborhoods are more popular than
others. More quantitatively, plotting number of photos of each location as a
function of their rank (Figure~\vref{f:pdis}), we notice that it first follows
a power law and after some point, a more abrupt one. Moreover, the same
phenomena occurs for other grid size, albeit with different $\alpha$
coefficients. Despite this similar behavior, it was more tricky to explicitly
exclude parts of the city.

\clearpage
\newgeometry{left=0.8cm,right=0.8cm,bottom=1cm,top=0.5cm}
\thispagestyle{empty}
\begin{figure}[p]
        \centering
		\begin{subfigure}[b]{0.5\textwidth}
			\includegraphics[height=.4\textheight,width=\textwidth,keepaspectratio]{distag}
			\caption{Tag distribution log-log scale over three regions of
			different scale.\label{f:tags}}
		\end{subfigure}~
        \begin{subfigure}[b]{0.5\textwidth}
			\includegraphics[height=.4\textheight,width=\textwidth,keepaspectratio]{../prez/pdistrib.png}
			\caption{Spatial distribution of photos over three grids with
			different granularity.\label{f:pdis}}
        \end{subfigure}

		\begin{subfigure}[b]{\textwidth}
			\centering
			\includegraphics[height=.6\textheight]{../heatmap.png}
			\caption{Photos count in logarithmic scale (the darker, the more
			photos).\label{f:heat}}
		\end{subfigure}
		\caption{.\label{f:distrib}}
\end{figure}
\restoregeometry

Let us return to one of our original problem, find which tags describe a given
location. The first approach would to filter this \numprint{1874} tags to keep
only those that are enough concentrated at one position and reject those that
too uniformly distributed. After that, it would simply be a matter of
returning those that appear in the place of interest. An example of this two
kind of tags are \textsf{museum} and \textsf{street}. As shown in
Figure~\vref{f:sm}, \textsf{museum} photos are mostly located around five or
six points whereas \textsf{street} is more diffuse. But instead of looking at
a map, we want a numerical statistic that allow us to distinguish between this
two cases.

\begin{figure}[hbtp]
	\centering
	\includegraphics[width=\columnwidth]{../prez/sm.png}
	\caption{Red dots denote photos tagged \textsf{museum} while blue ones are
	\textsf{street}.\label{f:sm}}
\end{figure}
