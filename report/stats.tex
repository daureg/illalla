\begin{figure}[hbtp]
\includegraphics[width=\columnwidth]{distag}
\caption{Tag distribution log-log scale over three regions of different scale.\label{f:tags}}
\end{figure}

The first thing was to look at the tags to get a sense of them. In San
Francisco after the preprocessing, there was a total of \numprint{4977625}
occurrences of the \numprint{145242} unique tags. But their distribution varies
widely, between the most popular one, \textsf{sanfrancisco}, used
\numprint{373427} times and the \numprint{101361} that are used less than 5
times. Some of them are shown in Table~\vref{tab:tags} but a more synthetic
visualization is presented in Figure~\vref{f:tags}, where we can see that like
words in a written documents, tags follows a power law.

\begin{table}[ht]
	\centering
	\begin{tabular}{lll}
		\toprule
		First 15 tags 	 & between 100 and 1000 & after \numprint{90000} \\
		\midrule
		sanfrancisco 	 & 2013 							   & sfgiantsfan \\
		california       & pacific                             & rolexbigboatseries \\
		iphoneography    & february                            & proshowgold \\
		square           & foundinsf                           & neutraface \\
		% squareformat     & dolorespark                         & natur€ \\
		instagramapp     & japaneseteagarden                   & lusty \\
		unitedstates     & boat                                & lightousetender \\
		sf               & 5k                                  & jennyholzer \\
		usa              & national                            & img0562jpg \\
		% ca               & σανφρανσίσκο                        & djguyruben \\
		san              & cruise                              & cutebaby \\
		francisco        & above                               & cardamine \\
		goldengatepark   & july2009                            & californiaproduce \\
		2010             & effortlesslyuploadedbymyeyeficard   & aroundwithb1 \\
		iphone           & dayofdecision                       & aquateenhungerforcemooninite \\
		\bottomrule
	\end{tabular}
	\caption{A sample of San Francisco tags, depending of their rank.\label{tab:tags}}
\end{table}

After making these observations, I decided to consider only tags with enough
support, both to ease the computational effort and to avoid outliers. For each
tags, I compute three simple metrics, total count, distinct users count and
time span. Using the three threshold (150 photos, 25 users, 500 days), I kept
only \numprint{1874} tags. It may seem quite restrictive but they still cover
68.6\% of all occurrences and we can always change these thresholds
later\footnote{For instance, with $(20, 2, 0)$, we get \numprint{12959} tags
and 84.4\% coverage.}.

We can then conduct a similar analysis over the locations in which photos
appear. Because of their large number, it was not convenient nor readable to
display them individually. Therefore, I discretized space as a regular grid of
size 200 by 200. In that case, each rectangular cell is around $80\times 70$
meters and I used this same method for all other spatial computation. A
natural way of visualizing them is to draw a heatmap (Figure~\vref{f:heat}).
We notice again that they are far from being uniformly distributed and that
some neighborhood are more popular than others. More quantitatively, number of
photos of each location as a function of their rank (Figure~\vref{f:pdis}), we
notice that it first follows a power law and after some point, a more
abrupt one. Moreover, the same phenomena occurs for other grid size, albeit
with different $\alpha$ coefficient. Despite this similar behavior, it was
more tricky to explicitly exclude parts of the city.

\begin{figure}[hbtp]
\includegraphics[width=\columnwidth]{../heatmap.png}
\caption{Photos count in logarithmic scale (the darker, the more
photos).\label{f:heat}}
\end{figure}

\begin{figure}[hbtp]
\includegraphics[width=\columnwidth]{../prez/pdistrib.png}
\caption{Spatial distribution of photos over three grids with different
granularity.\label{f:pdis}}
\end{figure}

Let us return to one of our original problem, find which tags describe a given
location. The first approach would to filter this \numprint{1874} tags to keep
only those that are enough concentrated at one position and reject those that
too uniformily distributed. After that, it would simply a matter of returning
those that appear in the place of interest. An example of this two kind of
tags are \textsf{museum} and \textsf{street}. As shown in Figure~\vref{f:sm},
\textsf{museum} photos are mostly located around five or six points whereas
\textsf{street} is more diffuse. But instead of looking at a map, we want a
numerical statistic that allow us to distinguish between this two cases.

\begin{figure}[hbtp]
\includegraphics[width=\columnwidth]{../prez/sm.png}
\caption{Red dots denote photos tagged \textsf{museum} while blue ones are
	\textsf{street}.\label{f:sm}}
\end{figure}
