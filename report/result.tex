With this last statistic, we will now see how we can answer the three initial
questions, albeit not optimally.

\subsection{Tag positioning}

To find where a tag appears, we simply compute its discrepancy over all
regions that satisfy some size constraints and we keep track of the $K$ ones
with the highest discrepancy. When this is done, we merge some of the
overlapping regions and discard the others. Namely, we rank region by
discrepancy, start from the top one, merge it with its two highest neighbors
and remove the others ones. Then we move to next cluster until we have visited
all regions once.

While this is conceptually easy, the main drawback is the sensitivity to the
scale of the grid and even for the same grid, the choice of parameters.
Figure~\vref{f:mus} show result for \textsf{museum} and Figure~\vref{f:ggp}
for \textsf{goldengatepark}. Because they cover area of small and larger size,
we see that the appropriated grid is different. The problem is that we need to
find the correct one before starting the computations.

\newgeometry{left=0.3cm,right=0.3cm,bottom=0cm,top=0.0cm}
\begin{figure}
        \centering
        \begin{subfigure}[b]{0.38\textwidth}
                \includegraphics[width=\textwidth]{mus/museum_200_130_2000_2_5.png}
                \caption{$g=200$, $K=1000$, $T=130$, $min=2$, $max=5$}
        \end{subfigure}~
        \begin{subfigure}[b]{0.38\textwidth}
                \includegraphics[width=\textwidth]{mus/museum_80_250_2000_1_4.png}
                \caption{$g=80$, $K=1000$, $T=250$, $min=1$, $max=4$}
        \end{subfigure}~

        \begin{subfigure}[b]{0.38\textwidth}
                \includegraphics[width=\textwidth]{mus/museum_200_250_2000_2_5.png}
                \caption{$g=200$, $K=1000$, $T=250$, $min=2$, $max=5$}
        \end{subfigure}~
        \begin{subfigure}[b]{0.38\textwidth}
                \includegraphics[width=\textwidth]{mus/museum_20_350_3000_1_1.png}
                \caption{$g=20$, $K=1000$, $T=350$, $min=1$, $max=1$}
        \end{subfigure}~

        \begin{subfigure}[b]{0.38\textwidth}
                \includegraphics[width=\textwidth]{mus/museum_200_300_1000_2_2.png}
                \caption{$g=200$, $K=1000$, $T=300$, $min=2$, $max=2$}
        \end{subfigure}~
        \begin{subfigure}[b]{0.38\textwidth}
                \includegraphics[width=\textwidth]{mus/museum_20_350_3000_1_4.png}
                \caption{$g=20$, $K=1000$, $T=350$, $min=1$, $max=4$}
        \end{subfigure}
        \caption{Photos with tags \textsf{museum} are blue dots and red
rectangles represent high discrepancy regions. $g$ is the subdivision of the
grid, $K$ the maximum number of regions before merging, $T$ the minimum
number of photos in a region to be considered, and $min$ and $max$ are the
size constraints of the regions in number of cells.\label{f:mus}}
\end{figure}
\begin{figure}
        \centering
        \begin{subfigure}[b]{0.38\textwidth}
                \includegraphics[width=\textwidth]{ggp/ggp_20_250_2000_1_5.png}
                \caption{$g=20$, $K=1000$, $T=250$, $min=1$, $max=5$}
        \end{subfigure}~
        \begin{subfigure}[b]{0.38\textwidth}
                \includegraphics[width=\textwidth]{ggp/ggp_80_200_1000_1_4.png}
                \caption{$g=80$, $K=1000$, $T=200$, $min=1$, $max=4$}
        \end{subfigure}~

        \begin{subfigure}[b]{0.38\textwidth}
                \includegraphics[width=\textwidth]{ggp/ggp_20_250_2000_1_6.png}
                \caption{$g=20$, $K=1000$, $T=250$, $min=1$, $max=6$}
        \end{subfigure}~
        \begin{subfigure}[b]{0.38\textwidth}
                \includegraphics[width=\textwidth]{ggp/ggp_200_100_2000_3_7.png}
                \caption{$g=200$, $K=1000$, $T=100$, $min=3$, $max=7$}
        \end{subfigure}~

        \begin{subfigure}[b]{0.38\textwidth}
                \includegraphics[width=\textwidth]{ggp/ggp_20_250_2000_1_7.png}
                \caption{$g=20$, $K=1000$, $T=250$, $min=1$, $max=7$}
        \end{subfigure}~
        \begin{subfigure}[b]{0.38\textwidth}
                \includegraphics[width=\textwidth]{ggp/ggp_200_200_1000_1_5.png}
                \caption{$g=200$, $K=1000$, $T=200$, $min=1$, $max=5$}
        \end{subfigure}
		\caption{Same as Figure~\ref{f:mus} but for
		\textsf{goldengatepark}.\label{f:ggp}}
\end{figure}
\restoregeometry

\subsection{Location describing}

We can also use discrepancy to describe a location as follows: we perform the
merging process just presented for all supported tags. For each of them, we
will thus obtain a few regions with an associated scalar value. Given a query
region, we go through this list and return matching tags, sorted by
discrepancy. Screenshots of a working demonstration are displayed in
Figure~\vref{f:loc}. One issue they do not show is that the list of results is
sometimes very sensitive to the rectangle selected. But they exhibit the
another one; this list can be very long and it would be desirable to limit it
(for instance in Alcatraz~(\ref{f:alca}), return only tags with discrepancy
above $3$). The matching process can also be improved, maybe by scaling the
discrepancy by the overlapping area between the query and the tag regions.

\begin{figure}[b]
\centering
\vspace{-1\baselineskip}
\includegraphics[height=0.40\textheight]{../adv3/tourist.png}
\caption{The $20$ more discrepant tags are their characteristics
location.\label{f:cover}}
\vspace{-1\baselineskip}
\end{figure}

\subsection{Map covering}

Finally, as a proxy for finding interesting locations, we tried to pave the
map with suitable tags. More precisely, we selected the highest discrepancy
region of each tag, ranked them and return the top ones. Result is shown in
Figure~\vref{f:cover}. In its current form, the visualization quickly become
unreadable if we allow overlapping tags. But there are more fundamental
issues. Namely, because it was computed with the values from the $200\times
200$ grid and regions smaller than $4$ cells at most, it misses larger areas
like the Golden Gate Park.  Moreover, discrepancy is an arguable criterion of
interest. Even if at first look, it seems to return mostly relevant locations,
it lacks flexibility and will always yield the same result, regardless of the
user typology and its preferences.

\newgeometry{hmargin=1.5cm}
\begin{figure}[p]
        \centering
        \begin{subfigure}[b]{0.49\textwidth}
                \includegraphics[width=\textwidth]{czoo.png}
                \caption{San Francisco Zoo, where Google Map indeed reports
the existence of a \enquote{Penguin Island}.}
        \end{subfigure}~
        \begin{subfigure}[b]{0.49\textwidth}
                \includegraphics[width=\textwidth]{cloh.png}
                \caption{California Palace of the Legion of Honor, a museum
					with many Auguste Rodin's sculptures.}
        \end{subfigure}

        \begin{subfigure}[b]{0.5\textwidth}
                \includegraphics[width=\textwidth]{calca.png}
                \caption{Alcatraz Island, aka \enquote{The Rock}, home of an
abandoned prison.\label{f:alca}}
        \end{subfigure}
		\caption{Description of three points of interest in San Francisco.\label{f:loc}}
\end{figure}
\restoregeometry
