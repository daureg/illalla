\begin{abstract}
Provide a short description of the project (problem studied, contributions, findings).
\end{abstract}

\section{Introduction}

A common way of organizing information on websites it to rely on tags provided
by user. Although these words are chosen freely, we do not expect them to be
arbitrary and thus hope being able to extract limited semantics from them. In
this project, we focus on tags associated with geolocated photos extracted
from
Flickr\footnote{\href{https://secure.flickr.com/}{http://www.flickr.com/}}.
Using this crowdsourced set of (tags, location) pairs, we investigate two
problems:
\begin{itemize}
	\item Given a tag, find the places in a city or in country where it is
		significantly concentrated.
	\item Conversely, given a location, find the tags that best describe it
		compared with other locations of similar scale.
\end{itemize}

In the first case, it could be applied in a touristic context. When arriving in a
new city, a person interested in \textsf{baseball} or \textsf{streetart} could
find relevant places according to the experience of inhabitants and previous
visitors. The second answer could be used by Flickr to suggest relevant tags
when users upload photos. Consequently, more sensibly tagged photos allow more
relevant search result and improve the website user experience.

Other applications may include inferring missing information (like location or
time of the day) from the tags or…

What are the contributions of this project?



\section{Setting}
\label{sec:setting}

Provide information that will be necessary for a reader to understand your contributions.

\subsection{General Description of the Dataset}
Using the Flickr API, we download metadata from every photos satisfying a set
of criteria: they contain at least one tag, they are located (with enough
precision, namely what Flickr calls street level), they have been uploaded
after January 1\textsuperscript{st}, 2008 and they belong to predefined
rectangular region. Most of the work was done on a set of around
\numprint{780000} photos in the city of San Francisco, but we also get data in
a part California\footnote{from San José to Reno} and over the whole United
States.

More precisely, in addition to tags and location, we know when each photos was
taken and uploaded, by which user and what title was given to it (the title
was not used except when it contains hashtag, that were converted to tags). A
typical data point look like this:

\begin{verbatim}
loc   : {type : Point,
         coordinates : [-122.39094, 37.7777]}
taken : ISODate(2013-02-03T15:59:49Z)
user  : 37996593020@N01
tags  : [geotagged,
         giants,
         sanfranciscogiants,
         gigantes,
         worldchampions]
title  : Champions
upload : ISODate(2013-02-05T18:36:49Z)
_id    : 8447225241
\end{verbatim}

\subsection{Useful Data Statistics}

\begin{figure}[hbtp]
\includegraphics[width=\columnwidth]{distag}
\label{fig:distag}
\caption{Tag distribution in the three regions}
\end{figure}

tag distrib => preprocessing >= after
photos distribution on a map
museum vs street, find difference
entropy and nKentropy
plot of pairwise dst and gravity dst
plot of mu,sigma grav
\section{Algorithms}
\label{sec:algorithms}

Define the algorithms you employ for the purposes of the project.

\section{Experimental Evaluation}

Here you report your findings from running your algorithms (Section
\ref{sec:algorithms}) on your data (Section \ref{sec:setting}).

\section{Conclusion \& Future Work}

\balance
Summarize your findings. Describe how you would extend your work.




