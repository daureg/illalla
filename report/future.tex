By \keyfinding{summarizing} what we have done so far, we will be able to see
which \keyfinding{limitations} can be addressed in further work. After
extracting photos metadata from Flickr and exploring their location and their
tags, we computed some statistics about them and notably the discrepancy of
the most supported tags. We then use this information to highlight various
relationships between tags and locations. But how can these results can be
improved?

\medskip

The first point to look at is \keyfinding{the data}. As mentioned on
page~\pageref{p:data}, they come with some noise and we expect the results to
be more relevant if we manage to \keyfinding{clean them}. Yet it remains to be
seen if it is worth the extra effort. A simpler direction would be to
\keyfinding{increase their amount}. This can be done by crawling photos from
others cities, to see if the same methods produce the same results or if
working only with San Francisco have introduced some bias. Another approach
would be to use more diverse sources like instagram, foursquare or any other
social networks that allows users to share localized and tagged contents.


The second limitation that arise from focusing on a single area is that we
were not confronted enough with one fundamental issue of spatial analysis, the
\keyfinding{modifiable areal unit problem}\cite{scale}, which states that
statistical measurements can be profoundly affected by the choice of the
spatial partition and its characteristics size (in our case, the grid and its
cells dimension).  It is an open question whether the current approach can
simply be tuned to tackle different area like states, countries or even the
whole world, or if it needs radical change and scale invariant statistics.

A related issue is that the presented approach comes with \keyfinding{many
parameters and thresholds}. That would be fine if they encode some prior
knowledge but the truth is, they were chosen empirically. Indeed, the problem
is mostly unsupervised. For instance, except for specific tag like the name of
buildings, most of them can be found in several locations. Likewise, places
can be described by many tags, maybe of various relevance but with no clear
cut. Finally, finding interesting locations is rather subjective and,
depending of the chosen criterion, can yield diverse results. The rather rigid
grid division is therefore quite limited, not least because not all locations
are rectangular! It may thus be more appropriate to try some
\keyfinding{unsupervised clustering methods}, for instance based on the
Euclidean norm in the case of the photos locations or drawn from natural
language \keyfinding{topic modeling} approaches in the case of
tags\cite{topicModel}.


Even with this lack of definitive results, it would be helpful to have a way
to \keyfinding{evaluate them}. One way could be to generate
\keyfinding{artificial data} from which we know precisely what should be found
inside, but it would again require some assumptions of our part. We can also
create \keyfinding{hand picked ground truth} by manually annotating tags and
locations, as in \cite{Rattenbury2009}. To cope with ambiguity, it would be
even better to aggregate evaluations from several individuals\footnote{But
also more time consuming, although this may be presented as a game, like
GeoGuessr: \url{http://geoguessr.com/}.}. An easier alternative would be to
manually extract famous landmarks from existing tourist guides and assess
precision and recall of the method.

\medskip

Finally, whereas photos come with \keyfinding{four kinds of
information}---tags, location, user and time---we only work with the first two
and it would undoubtedly be more interesting to \keyfinding{use all of them}.
Let us try to list systematically what we can do of it.

First, by focusing on a single one and building a profile using the rest, we
may devise a \keyfinding{similarity measure and use it to perform clustering}.
In the case of users, it may divide between wealthy and not, young and old,
male or female \cite{gender}, tourist or inhabitant or richer category like
family with kids and traveler of a package tour. As said above, tags can be
grouped into common topic\footnote{Or at first, we can look at
co-occurrences.}. Lastly, although times and locations are subset of
$\mathbb{R}$ and $\mathbb{R}^2$ and as such already have a natural metric,
they carry additional semantic that we would like to exploit. For instance, we
assume that by periodicity, Sundays in 2008 and 2013 share common patterns
although they are far apart in time. Likewise, maybe locations in front of the
water or that consist of park have similar characteristics.

Then we should consider \keyfinding{pair of concepts}. The present work deals
with tags \texrel locations in both directions, thus there are $\binom{4}{2} -
1$ others to look for: tags \texrel time, tags \texrel users, locations \texrel
times, locations \texrel users and times \texrel users. Yet all of them may
not be interesting and we need to hierarchize our priorities. 
