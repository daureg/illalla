\documentclass[a4paper,11pt,draft]{scrartcl}
\usepackage[vmargin=1.9cm]{geometry}
\usepackage{amsmath}
\usepackage{amsfonts}
\usepackage{amssymb}
\usepackage{microtype}
\DeclareMathOperator*{\argmax}{argmax}
\DeclareMathOperator*{\argmin}{argmin}
\renewcommand{\equiv}{\Leftrightarrow}
\newcommand{\vnorm}[1]{\left|\left|#1\right|\right|}
% \usepackage[style=ieee,sorting=nyt,backend=biber,defernumbers=true]{biblatex}
% \addbibresource{../../aalto/thesis.bib}
% \usepackage[svgnames]{xcolor}
% \usepackage{hyperref}
% \hypersetup{% draft,
%     colorlinks=true, linktocpage=true, breaklinks=true, pdfpagemode=UseNone,
%     plainpages=false, bookmarksnumbered, hypertexnames=true, pdfhighlight=/O,
%     urlcolor=Purple, linkcolor=RoyalBlue, citecolor=LimeGreen}
\newcommand{\autocite}[1]{}
\newcommand{\textcite}[1]{}
\title{Neighborhood matching}
\begin{document}
\maketitle
\section*{Setup}
We have two cities, $C_1$ and $C_2$. In each of them lie objects of different
kind (in our case: photos, check-ins and venues). These objects have a
position in $\mathbb{R}^2$, along with other attributes like time, tags,
category or number of visitors.

\section*{Goal}
As an input, we are given a polygonal region $R$ contained in $C_1$.
% (this could be rectangles in the beginning and later more complicated, yet
% reasonable, shape)
The goal is then to find the region $R'$ in $C_2$ that is the most similar to
$R$.
% (with a potential warning if no suitable candidate is found).
Now let us see \emph{one} way to turn this objective into a problem.

\section*{Formalization}

First we need a function that transforms a region $R \in \mathcal{R}$, defined
by all the $e$ objects it contains, into a numeric representation. In the
simplest case, we aggregate information from these $e$ objects into a single
$d$-dimensional feature vector:
\begin{align*}
	\phi \colon \mathcal{R} &\to \mathbb{R}^d \\
	R &\mapsto x
\end{align*}

But this aggregation loses information. Thus a more appealing alternative
would be to keep track of the $e'$ main objects (the venues) and to represent
them individually. In addition, each region have a set of $g$ global features
that are not associated with any specific objects. This define a more flexible
$\phi$:
\begin{align*}
	\phi \colon \mathcal{R} &\to
	\mathbb{R}^{e'\times d} \times \mathbb{R}^g = \mathcal{F} \\
	R &\mapsto (X, x)
\end{align*}

Then we need a function, parametrized by $\theta$, to assess the distance
between two regions in different cities
\[
	\delta_{\theta} \colon \mathcal{F} \times \mathcal{F} \to \mathbb{R}^+
\]

\medskip

At this stage, we can fulfill our objective by computing, for a given $R \in
\mathcal{R}(C_1)$
\begin{equation}
    \argmin_{R' \in \mathcal{R}(C_2)}\; \delta_{\theta}
    \left(\phi(R), \phi(R')\right)
    \label{e:onetheta}
\end{equation}

A more refined answer would acknowledge that not all regions can be compared in
the same way and thus solve
\begin{equation}
    \argmin_{\theta ,\, R' \in \mathcal{R}(C_2)}\; \delta_{\theta}
    \left(\phi(R), \phi(R')\right) + cost(\theta)
    \label{e:manythetas}
\end{equation}
where the regularisation of $\theta$ ensures that there is still a common
ground in the way similarity is computed.

\section*{Point Pattern Matching}

\eqref{e:manythetas} can be casted as a \emph{Point Pattern
Matching}\footnote{Also known as \emph{Point set registration}.} problem.
Following \textcite{PointPatternMatching08}, let $P$ be the $e'$ points of
$\mathbb{R}^d$ that make up $R$, and let $T$ be the $n$ points of
$\mathbb{R}^d$ lying in $C_2$. The problem asks whether there is a subset $I
\subseteq T$ such that $\delta_{\theta}\left(P, I\right) \leq \epsilon$. In
this case, $\theta$ represents a linear transformation applied to $P$. And
instead of explicit regularisation, it can only be drawn from a fixed class of
transformations. Another constraint is that $I$ must be restricted to subsets
of venues that indeed form neighborhood.

\section*{Choosing a relevant $\delta$}

\begin{itemize}
    \item There are different distances between two point sets. For instance,
        the bottleneck distance\autocite{Bottleneck96} (the smallest distance
        between pairs of farthest point), the sum of the distance of each
        point to its closest neighbor, or the Hausdorff distance (the largest
        distance between pairs of closest point) and its
        variation\autocite{ModifiedHausdorff94}. From what I have seen, these
        measures have been used mostly in 2 or 3 dimensional space.  Thus it
        would be interesting to see how they behave in higher dimension.
    \item We also have to take global feature into account, for instance using
        histogram distance\autocite{HistogramDistance02} to compare the time
        activity of different regions.
    \item Weighting venues, either explicitly or, more likely, implicitly (for
        instance by their relative popularity in the region) will provide more
        flexibility. Especially, a fixed fraction of venues may have zero weight
        because we consider they don't really belong to a given neighborhood.
    \item Instead of a geometric approach, this could also be tackled by
        information theoretic distance over some probability distributions,
	like the Kullback--Leibler divergence. Or as suggested by the paper
	from Arto\autocite{InfoMatching10}, by looking at the mutual
	information between two regions.
    \item Finally, we could build a graph out of venues. It may for instance
        contains information about their relation in the original 2D space. Or
        be tailored to take categories into account\autocite{fitting09}. Then
        they can be compared using appropriate
        similarity\autocite{GraphSimilarity14}. The drawback is that it seems
        somewhat artificial and contrived.
\end{itemize}
% \printbibliography
\end{document}
